%------------------------------------
% Template based on the version by:
% Dario Taraborelli
% URL: http://nitens.org/taraborelli/cvtex
% Some rights reserved: http://creativecommons.org/licenses/by-sa/3.0/
%------------------------------------


\documentclass[11pt, a4paper]{article}
\usepackage[spanish,french,english]{babel}
%\babelhyphenation[french][in-for-ma-tion]
\usepackage{fontspec} 
\usepackage{graphicx}
\usepackage{tikz}
\usepackage{color, colortbl}
\usepackage{enumitem}
\usepackage{xurl}

% DOCUMENT LAYOUT
\usepackage{geometry} 
\geometry{a4paper, textwidth=5.5in, textheight=9.1in, marginparsep=7pt, marginparwidth=.6in}
\setlength\parindent{0in}

% FONTS
\usepackage{xunicode}
\usepackage{xltxtra}
\defaultfontfeatures{Mapping=tex-text} % converts LaTeX specials (``quotes'' --- dashes etc.) to unicode
\setromanfont [Ligatures={Common}, BoldFont={Gentium Basic Bold}, ItalicFont={Gentium Basic Italic}]{Gentium Basic}
%\setmonofont[Scale=0.8]{Monaco} 
% ---- CUSTOM AMPERSAND
\newcommand{\amper}{{\fontspec[Scale=.95]{Gentium Basic Italic}\selectfont\itshape\&}}
% ---- MARGIN YEARS
\usepackage{marginnote}
\newcommand{\years}[1]{\marginnote{\scriptsize #1}}
\renewcommand*{\raggedleftmarginnote}{}
\setlength{\marginparsep}{7pt}
\reversemarginpar

\newcommand{\ind}[1]{\textcolor{black!62!white}{\selectfont#1}}

% HEADINGS
\usepackage{sectsty} 
\usepackage[normalem]{ulem} 
\sectionfont{\rmfamily\mdseries\upshape\Large}
\subsectionfont{\rmfamily\mdseries\upshape\large}
\subsubsectionfont{\rmfamily\bfseries\upshape\normalsize} 
%\subsubsectionfont{\rmfamily\mdseries\upshape\normalsize} 

%BREAKING DOIs

\usepackage{expl3}
\ExplSyntaxOn
\newcommand \breakDOI[1]
 {
  \tl_map_inline:nn { #1 } { \href{http://dx.doi.org/#1}{##1} \penalty0 \scan_stop: }
 }
\ExplSyntaxOff


% PDF SETUP


\makeatletter
\renewcommand\@seccntformat[1]{}
\makeatother

\usepackage[dvipdfm, bookmarksopen=true, bookmarks=true, colorlinks, breaklinks=true, pdftitle={Helena Bermúdez Sabel - vita},pdfauthor={Helena Bermúdez Sabel}]{hyperref}  
\hypersetup{linkcolor=blue!70!black,citecolor=blue!70!black,filecolor=black,urlcolor=blue!70!black} 



\usepackage[open,openlevel=2]{bookmark}

% DOCUMENT
\begin{document}
{\LARGE Helena Bermúdez Sabel}\\[1cm]
Institut des sciences du langage\\
Université de Neuchâtel\\Pierre-à-Mazel 7\\2000 Neuchâtel \\ Switzerland\\[.2cm]
%Phone: \texttt{+41 021 692 47 02}\\[.2cm]
email: \href{mailto:helena.bermudez@unine.ch}{helena.bermudez@unine.ch}\\
ORCID: \href{https://orcid.org/0000-0002-8627-1367}{0000-0002-8627-1367}\\
Website: \href{https://helenasabel.github.io/}{https://helenasabel.github.io/} 



\section{Profile}
\begin{itemize}[noitemsep]
 \item Researcher with eight years of experience in different projects at the intersection of Humanities and Information Technologies, with a focus on Natural Language Processing (NLP) Applications.
 \item Academic background in Philology and Medieval Studies with a strong foundation in Romance Linguistics and Latin.
 \item Strong programming skills. See Github profile at \href{https://github.com/helenasabel}{https://github.com/helenasabel} and  “\hyperref[DHprojects]{Digital Humanities Projects}” below.
 \item Expertise in digital editing and the creation of interfaces for philological textual analysis aided by NLP tools.
 \item Participation in international and interdisciplinary teams; elected member of the Text Encoding Initiative's Technical Council (TEI-C).
 \item Active involvement in dissemination and training activities related to Digital Humanities.
 \item Student supervision experience at postgraduate level (MA and PhD).
\item Experience in group management and coordination of research activities.
\item Day-to-day involvement in collaborative works through which I have developed good interpersonal and teamwork skills.
\end{itemize}


%%\hrule
\section{Current position}
\years{01/2021--}Postdoctoral researcher (\emph{Chercheuse FNS senior}), Institut des sciences du langage, Université de Neuchâtel.
% \textcolor{black!62!white}{
% \begin{itemize}[noitemsep,topsep=0pt]
%  \item Design, development and implementation of all computer programs for the project \textit{A World of Possibilities. Modal pathways over an extra-long period of time: the diachrony of modality in the Latin language}
% \item Responsible for data retrieval, data administration and data processing for linguistic analysis
% \item Design and development of the NLP-assisted annotation pipeline 
% \item Formalisation of annotation schemas
% \item Linguistic annotation in Latin (both automatic and manual annotation)
% \item Frontend and backend development
% \item Server maintenance (Linux)
% \end{itemize}}


%%\hrule
\section{Areas of specialization}
Romance and Latin Linguistics. Corpus linguistics. Digital Humanities. Digital Philology. Textual criticism.

%%\hrule
\section{Appointments held}
\noindent
\years{2019--2020}Postdoctoral researcher, Université de Lausanne, Lausanne (Switzerland)\\
%   \ind{Same activities as in current position (the project continues in Neuchâtel).}\\

\years{2017--2019}Researcher, Universidad Nacional de Educación a Distancia, Madrid (Spain)
% \textcolor{black!62!white}{
% \begin{itemize}[noitemsep,topsep=0pt]
%  \item Development of a semantic model for European Poetry in the Semantic Web ecosystem. For this task, I analysed numerous online resources which contained poetry information related to European traditions.
%  \item Assistance as a domain expert in both Linguistics and Literature in the development of NLP tools. 
%  \item Consultant for research projects dealing with the publication of digital editions and/or textual databases.
%  \end{itemize}
%  }

\years{2012--2016}Research Assistant, Universidade de Santiago de Compostela (Spain)
% \textcolor{black!62!white}{
% \begin{itemize}[noitemsep,topsep=0pt]
%  \item Undergraduate teaching (mostly focused on technological resources for the study of Medieval Language and Literature).
%  \item Development of individual research projects based on the application of computational methods for the study of Galician-Portuguese language and literature.
%  \end{itemize}
%  }
 
%\hrule
\section{Education}
\noindent
\subsection{Academic degrees}
\years{01/2019}PhD in Medieval Studies, Universidade de Santiago de Compostela\\
\ind{\textbf{Thesis title:} \textit{As Humanidades Digitais e a sua aplicação à variação linguística na lírica galego-portuguesa} (Linguistic variation in Galician-Portuguese lyric: a Digital Humanities approach).}\\
Supervisors: Pilar Lorenzo Gradín \amper{} Mariña Arbor Aldea\\
Committee: Mercedes Brea (Universidade de Santiago de Compostela), José Manuel Lucía Megías (Universidad Complutense de Madrid, Spain), Elena Pierazzo (Université de Tours, France)\\
Evaluation: “sobresaliente” \textit{cum laude}\\
\years{02/2014}BA in Portuguese Philology, Universidade de Santiago de Compostela\\
\years{07/2012}MA in European Medieval Studies: Images, Texts and
Contexts, Universidade de Santiago de Compostela\\
\years{07/2011}BA in Romance Philology, Universidade de Santiago de Compostela
%\hrule
\subsection{Specialised and professional training}
\years{2019}\textit{Journée de Initiation à la lemmatisation des textes médievaux}. June 17, 2019. IRHT-CNRS (Paris, France).\\
\years{2019}\textit{LiLa Workshop: Linguistic Resources \amper{} NLP Tools for Latin}. June 3--4, 2019. LiLa Project (Milan, Italy).\\
\years{2017}\textit{Second Summer Datathon on Linguistic Linked Open Data} (SD-LLOD-17). June 26--30, 2017.
Ontology Engineering Group, Universidad Politécnica de Madrid (Cercedilla, Spain).\\
\years{2016}\textit{Diploma en docencia universitaria}. Universidade de Santiago de Compostela \ind{(more than 100 hours of specialised training)}.\\
\years{2015}\textit{Computer-supported collation with CollateX}. June 29, 2015. Alliance of Digital Humanities Organizations  (Sydney, Australia).\\
\years{2014}\textit{Computational Methods in the Humanities}. 08/2014--12/2014. University of Pittsburgh (Pittsburgh, USA).
\\
\years{2014}\textit{Taking TEI further: transforming and publishing TEI}. March 11--13, 2014. Women Writers Project – Northeastern University (Boston, USA).

\years{2013}\textit{GRADSchool: competence training course}. October 14--17, 2013. CRAC – Vitae,  Fundación Barrié de la Maza (Spain).

\years{2013}\textit{Digital Humanities: Aproaches and Applications}. July 8--10, 2013. Humanidades Digitales Hispánicas (Spain).

\years{2013}\textit{Traning School in Codicology}. April 29--May 3, 2013. Medioevo Europeo — Medieval Cultures and Technological Resources (Berlin, Germany).

\subsection{Research stays}
 \years{03--04/2019} Institute of Czech Literature, Prague (Czech Republic), 1 month. \textit{Josef Dobrovský Fellowship}, funded by the Czech Academy of Sciences. \\
\years{01--04/2016} University of Pittsburgh, Pittsburgh (USA), 3 months. \textit{Short research stays abroad for university faculty in training} (Programa de Formación del Profesorado Universitario, FPU) funded by the Ministerio de Educación, Cultura y Deporte (Spanish Government)\\
\years{08--12/2014} University of Pittsburgh, Pittsburgh (USA), 4 months. \textit{Becas de Posgrado en el Extranjero} (Fundación Pedro Barrié de la Maza). \\
\years{03--05/2014} An Foras Feasa (University of Ireland, Maynooth), Maynooth (Ireland), 2 months.


% \subsection{Languages skills}
% \begin{itemize}[noitemsep]
%  \item Galician (first language)
%  \item Spanish (first language)
%  \item English: C2
%  \item Portuguese: C2
%  \item French: B2
%  \item Other languages: Occitan, Catalan, Italian, Modern Greek
% \end{itemize}


\subsection{Computer skills}
\begin{itemize}[noitemsep]
 \item XML technologies: XML, XML-TEI, XPath, XSLT, XQuery, Schematron, RELAX NG, KML, SVG
 \item General-purpose programming languages: Python
 \item Web technologies: HTML5, RDFa, CSS3, JavaScript, PHP
 \item Other: Turtle (Terse RDF Triple Language), Web Ontology Language (OWL), LaTeX
 \item Software: TXM, oXygen XML Editor, eXist DB, Cytoscape, Google Earth, Inkscape 
\end{itemize}



\section{Publications \& talks}

\subsection{Peer-reviewed scientific journals}
\noindent
\years{2022}McGillivray, B., Kondakova, D., Burman, A., Dell’Oro, F., \underline{Bermúdez Sabel, H.}, Marongiu, P. \amper{} M. Márquez Cruz. “A new corpus annotation framework for Latin diachronic lexical semantics”. \textit{Journal of Latin Linguistics} 21(1): 47--105,  \breakDOI{doi:10.1515/joll-2022-2007}. \\
\years{2022}\underline{Bermúdez Sabel, H.}, Díez Platas, M.L.,
Ros, S.  \amper{} E. González-Blanco, “Towards a common model for European Poetry: Challenges and solutions”, \textit{Digital Scholarship in the Humanities} 37(4): 921–933. \breakDOI{doi:10.1093/llc/fqab106}\\
\years{2021}Álvarez-Mellado, E., Díez, M.L., Ruiz-Fabo, P.,  \underline{Bermúdez, H.},
Ros, S.  \amper{} E. González-Blanco. “TEI-friendly annotation scheme for medieval named entities:
a case on a Spanish medieval corpus”. \textit{Language Resources \amper{} Evaluation}, 55: 525–549. \breakDOI{doi:10.1007/s10579-020-09516-2}.\\
\years{2020}Ruiz Fabo, P., \underline{Bermúdez Sabel, H.}, Martínez Cantón, C. \amper{} E. González-Blanco, “The Diachronic Spanish Sonnet Corpus: TEI and linked open data encoding, data distribution, and metrical findings”,
\textit{Digital Scholarship in the Humanities}, 36 (Supplement 1): i68–i80,  \breakDOI{doi:10.1093/llc/fqaa035}\\
%\ind{Indexed in Scopus, Social Sciences Citation Index®, Computer Science Index}
\years{2019}\underline{Bermúdez Sabel, H.}, “Digital Tools for Semantic Annotation: the WoPoss Use Case”, \href{https://www.unil.ch/files/live/sites/sli/files/shared/BIL/BIL30.pdf}{Bulletin de l’Institut de linguistique}, 30: 12–37.\\
\years{2019}\underline{Bermúdez Sabel, H.}, “Encoding of Variant Taxonomies in TEI”, \textit{Journal of the Text Encoding Initiative}, Issue 11 (June). \breakDOI{doi:10.4000/jtei.2676}.\\
%\ind{Indexed in Semantic Scholar, Google Scholar, ROAD. Directory of Open Access Resources}
\years{2019}Ruiz Fabo, P. \amper{} \underline{H. Bermúdez Sabel}, “Navegación de corpus a través de anotaciones lingüísticas automáticas obtenidas por Procesamiento del Lenguaje Natural: de anecdótico a ecdótico”, \textit{Revista de Humanidades Digitales}, 4: 136–161. \breakDOI{doi:10.5944/rhd.vol.4.2019.25186}\\
%\ind{(Indexed in ERIH PLUS, Google Scholar, Semantic Scholar, Worldcat)}\\
\years{2019}\underline{Bermúdez Sabel, H} \amper{} E. González-Blanco, “Prefacio”, \textit{Revista de Poética Medieval}, 33: 11–23. \href{http://hdl.handle.net/10017/43770}{http://hdl.handle.net/10017/43770}.\\
%\ind{Indexed in ERIH PLUS}\\
\years{2018}Triplette, S., Beshero-Bondar, E. \amper{} \underline{H. Bermúdez Sabel}, “A Digital Humanities Approach to Cultural Translation in Robert Southey’s Amadis of Gaul”, \textit{Journal of Translation Studies}, 2(1): 35–58.\\
%\ind{(Indexed in EBSCOhost)}\\
\years{2017}\underline{Bermúdez Sabel, H.}, “Colación asistida por ordenador: estado de la cuestión y retos”, \textit{Revista de Humanidades Digitales}, 1: 20-34. \breakDOI{doi:10.5944/rhd.vol.1.2017.16678} \\
%\ind{(Indexed in ERIH PLUS, Google Scholar, Semantic Scholar, Worldcat)}



\subsection{Peer-reviewed conference proceedings}
\noindent
\years{2022}\underline{Bermudez Sabel, H.}, Dell’Oro, F., Montrichard, C. \amper{} C. Rossari. “Setting Up Bilingual Comparable Corpora with Non-Contemporary Languages”. In R. Rapp \textit{et al.} (eds.), \textit{15th Workshop on Building and Using Comparable Corpora}. European Language Resources Association (ELRA), pp. 56–60. \href{https://aclanthology.org/2022.bucc-1.8}{https://aclanthology.org/2022.bucc-1.8}\\
\years{2022}Diez Platas, M. L., \underline{Bermúdez Sabel, H.}, Ros, S., González-Blanco, E., Corcho, Ó., Gómez, O. K., Hernández-Lorenzo, L., Sisto, M. D., Rosa, J. de la, Pérez, Á., Diez, A., \amper{} J. L. Rodriguez. “Description of Postdata Poetry Ontology V1.0”. In P. Plecháč \textit{et al.} (eds.), \textit{Tackling the Toolkit: Plotting Poetry through Computational Literary Studies}. ICL CAS, Prague, pp. 15–30, \breakDOI{doi:10.51305/ICL.CZ.9788076580336.02}\\
\years{2021}\underline{Bermúdez Sabel, H.},“Trobadores de corte en corte: Visualización dos centros culturais ibéricos
tradomedievais” Castiñeiras López, J. \amper{} M. Cendón Fernández (eds.), \textit{Viajeros de la Antigüedad al
Nuevo Mundo}. Santiago de Compostela: Universidade de Santiago de Compostela, Servizo de Publicacións, pp. 295–307.

\years{2020}Dell’Oro, F., \underline{Bermúdez Sabel, H.} \amper{} P. Marongiu. “Implemented to Be Shared: the WoPoss Annotation of Semantic Modality in a Latin Diachronic Corpus”. In \textit{Sharing the Experience: Workflows for the Digital Humanities. Proceedings of the DARIAH-CH Workshop 2019 (Neuchâtel)}.  \breakDOI{doi:10.5281/zenodo.3739439}

\years{2018}Curado Malta, M., \underline{Bermúdez Sabel, H.}, Baptista, A. A. \amper{} E. González-Blanco, “Validation of a metadata application profile domain model”, \textit{Proc. Int’l Conf. on Dublin Core and Metadata Applications 2018}, pp. 65–75. ISNN 1939-1366.
       \href{http://dcevents.dublincore.org/IntConf/dc-2018/paper/viewFile/555/675}{http://dcevents.dublincore.org/IntConf/dc-2018/paper/viewFile/555/675}\\
    %\ind{(Indexed in Scopus)}\\
\years{2018}Tittel, S., \underline{Bermúdez-Sabel, H.} \amper{} C. Chiarchos, “Using RDFa to Link Text and Dictionary Data for Medieval French”. In J. McCrae \textit{et al.} (eds.), \textit{Proceedings of the 6th Workshop on Linked Data in Linguistics (LDL-2016): Towards Linguistic Data Science}. European Language Resources Association (ELRA), Paris, France, Miyazaki, Japan. ISBN 979-10-95546-19-1 \\
%\ind{(Indexed in Publons)}
      \href{http://lrec-conf.org/workshops/lrec2018/W23/pdf/10\_W23.pdf}{http://lrec-conf.org/workshops/lrec2018/W23/pdf/10\_W23.pdf}\\
\years{2017}\underline{Bermúdez-Sabel H.}, Curado Malta M. \amper{} E. Gonzalez-Blanco, “Towards Interoperability in the European Poetry Community: The Standardization of Philological Concepts”. In J. Gracia \textit{et al.} (eds.), \textit{Language, Data, and Knowledge. LDK 2017.} Lecture Notes in Computer Science, vol 10318. Springer, Cham, pp. 156–165,  \breakDOI{doi:10.1007/978-3-319-59888-8\_14}\\
%\ind{(Indexed in Scopus, EI Engineering Index, Google Scholar, DBLP)}\\
\years{2017}Curado Malta, M., \underline{Bermúdez Sabel, H.} \amper{} E. González-Blanco, 
“Modelação semântica: o caso de modelação da poesia”. In A. Terra \amper{} M. Carvalho (eds.), 
\textit{Gestores de Informação para o século XXI}. Instituto Politécnico do Porto, Instituto Superior de Contabilidade e Administração do Porto, Porto, ISBN: 978-989-97851-3-7, pp. 32--44.


\subsection{Contributions to books}
\noindent
\years{2022}\underline{Bermúdez Sabel, H.}, Ruiz Fabo, P. \amper{} C. Martínez Cantón. “DISCOvering Spanish Sonnets: A Circular Reading Experience”. In Bories, A. \textit{et al.} (eds.), \textit{Computational Stylistics in
Poetry, Prose, and Drama}. Berlin: De Gruyter. [In press]\\
\years{2022}\underline{Bermúdez Sabel, H}. “L’édition numérique au service de la philologie matérielle.
Modèles de la lyrique galégo-portugaise”. In D. González (ed.), \textit{Verdades duplas. A verdade do texto e a verdade material. Cancioneiros e fragmentos galego-portugueses}. ArGaMed 5/2022. Santiago de Compostela: Centro Ramón Piñeiro para a Investigación en Humanidades, pp. 11--30, \href{http://www.cirp.gal/publicacions/pub-0570.html}{http://www.cirp.gal/publicacions/pub-0570.html}.\\
\years{2022}Dell’Oro, F., \underline{Bermúdez Sabel, H.} \amper{} P. Marongiu. “Pygmalion in the classroom: a tool to draw lexicographic diachronic maps and their application to didactics”. In M. Márquez Cruz \amper{} V. Ferreira Martins, \textit{La lexicografía didáctica: Reflexiones y recursos orientados al aprendizaje de lenguas}. Madrid: Guillermo Escolar Editor, pp. 35–48.\\
\years{2018}\underline{Bermúdez Sabel, H}, “Anotación multicamada externa e o enriquecemento de edicións dixitais”, \textit{in} González, D. \amper{} H. Bermúdez Sabel (eds.), \textit{Humanidades Digitales. Miradas hacia la Edad Media}. Berlin: De Gruyter, pp. 4--17, \href{https://doi.org/10.1515/9783110585421-002}{doi:10.1515/9783110585421-002}.\\
%\ind{(Indexed in JSTOR, WorldCat, Google Scholar, Semantic Scholar)}\\
\years{2018}Fernández Guiadanes, A. \amper{} \underline{H. Bermúdez Sabel}, “Da transcrición paleográfica ás bases de datos: Problemas e solucións na lírica galego-portuguesa”, \textit{in} González, D. \amper{} H. Bermúdez Sabel (eds.), \textit{Humanidades Digitales. Miradas hacia la Edad Media}. Berlin: De Gruyter, pp. 34--48,  \breakDOI{doi:10.1515/9783110585421-005}.\\
%\ind{(Indexed in JSTOR, WorldCat, Google Scholar, Semantic Scholar)}\\
\years{2016}\underline{Bermúdez Sabel, H.}, “Variación gráfica na lírica profana galego-portuguesa: \textit{T} vs \textit{B,V}”, \textit{in} Corral Díaz, E. \textit{et al.} (eds.), \textit{Cantares de amigos. Estudos en homenaxe a Mercedes Brea}. Santiago de Compostela: Universidade de Santiago de Compostela, Servizo de Publicacións, pp. 109--115.\\
%\ind{(Indexed in Dialnet, Semantic Scholar)}\\
\years{2015}\underline{Bermúdez Sabel, H.}, “A edición sinóptica e a súa aplicación ao estudo da variación lingüística na lírica galego-portuguesa”, \textit{in} Castro, O. \amper{} M. García Liñeira (eds.), \textit{Trama e urda: Contribucións multidisciplinares desde os estudos galegos}. Santiago de Compostela: Consello da Cultura Galega–Asociación Internacional de Estudos Galegos, pp. 99--115, \breakDOI{doi:10.17075/tucmeg.2015.006}. \\
%\ind{(Indexed in Dialnet, SPI)}
 
 
\subsection{Reviews}
\noindent
\years{2021}\underline{Bermúdez Sabel, H.}, “Reviewing the bread and butter of CoReMa, Cooking Recipes of the Middle Ages”, \textit{RIDE – A review journal for digital editions and resources}. Issue 14: Scholarly Editions. \breakDOI{doi:10.18716/ride.a.14.1}.


\subsection{Edited volumes}
\noindent
\years{2019}González-Blanco, E. \amper{} \underline{H. Bermúdez Sabel}, \textit{Revista de Poética Medieval}, 33, “Los repertorios poéticos digitales: del Medievo a la interoperabilidad”. ISSN: 1137-8905. \\ \href{http://hdl.handle.net/10017/43757}{http://hdl.handle.net/10017/43757}\\
%\ind{Indexed in ERIH PLUS}
\years{2019}Plecháč, P., Scherr, B., Skulacheva, T., \underline{Bermúdez-Sabel, H.} \amper{} R. Kolár (eds.), \textit{Quantitative approaches to versification}, Prague: Institute of Czech Literature of the Czech Academy of Sciences, ISBN 978-80-88069-83-6.\\
%\ind{Indexed in Web of Science}
\years{2018}González, D. \amper{} \underline{H. Bermúdez Sabel} (eds.), \textit{Humanidades Digitales. Miradas hacia la Edad Media}. Berlin: De Gruyter, \breakDOI{doi:10.1515/9783110585421}.\\
%\ind{(Indexed in JSTOR, Google Scholar, Semantic Scholar)}\\
%       Reviews: \\
%         \begin{itemize}
%            \item García-Fernández, Miguel (2019): “Déborah González y Helena Bermúdez Sabel (eds.), Humanidades digitales. Miradas hacia la Edad Media, Berlin; Boston, De Gruyter, 2019, 1.a edición, 259 pp., ISBN 978-3-11-058541-4, e-ISBN (PDF) 978-3-11-058542-1, e-ISBN (EPUB) 978-3-11-058555-1”, \textit{Cuadernos Medievales}, 26: 101–4.}
          %\end{itemize}

      
\subsection{Contributions to conferences}
\subsubsection*{Refereed conference talks}
\noindent
\years{2023}López Izquierdo, M. \amper{} \underline{H. Bermúdez Sabel}, “En busca de un índice sociolectal computable: propuesta metodológica”. \textit{XXIII. Deutscher Hispanistentag}. February 22--25, 2023. Graz (Austria).\\
\years{2023}Ortuño Casanova, R., Martínez Cantón, C., Ruiz Fabo, P. \amper{} \underline{H. Bermúdez Sabel}, “Reescribiendo la historia de la literatura hispanofilipina en \textit{DigiPhiLit}: un análisis de contenido y forma de sonetos modernistas”. \textit{XXIII. Deutscher Hispanistentag}. February 22--25, 2023. Graz (Austria).\\
\years{2022}Dell’Oro, F. \amper{} \underline{H. Bermúdez Sabel}, “A diachronic corpus to study modality in the Latin language : the WoPoss experience step by step”, \textit{La constitution de corpus en diachronie longue : Méthodologies, objectifs et exploitations linguistiques et stylistiques}. October, 13--14, 2022. Grenoble (France).\\
\years{2022}Montrichard, C., \underline{Bermúdez Sabel, H.}, Rossari, C. \amper{} F. Dell’Oro, “Interroger la modalité en latin et en français : construction, annotation et exploitation de corpus”, \textit{La constitution de corpus en diachronie longue : Méthodologies, objectifs et exploitations linguistiques et stylistiques}. October, 13–14, 2022. Grenoble (France).\\
\years{2022}Beshero-Bondar, E.,  Viglianti, R., \underline{Bermúdez Sabel, H.} \amper{} J. Jenstad, “Revising Sex and Gender in the TEI Guidelines”, \textit{TEI Conference and Members' Meeting 2022}. September 12–16, 2022. Newcastle (United Kingdom).\\
\years{2022}Bauman, S., \underline{Bermúdez Sabel, H.}, Holmes, M. \amper{} D. Maus, “atop: another TEI ODD processor”, \textit{TEI Conference and Members' Meeting 2022}. September 12–16, 2022. Newcastle (United Kingdom).\\
\years{2022}\underline{Bermúdez Sabel, H.}, Martínez Cantón, C. \amper{} P. Ruiz Fabo, “From poetry to song. A corpus-based approach to textual variation”, \textit{Plotting Poetry (and Poetics) 5}. July 4–6, 2022. Tartu (Estonia).\\
\years{2022}\underline{Bermúdez Sabel, H.}, Marongiu, P. \amper{} F. Dell’Oro. “Premodal, modal and postmodal: a corpus-based study of the polyfunctionality of Latin modal markers”. \textit{La postmodalité et les cycles de vie des expressions modales}. Maison de la Recherche en Sciences Humaines (MRSH), Université de Caen Normandie. June 2-3, 2022. Caen (France).\\
\years{2021}\underline{Bermúdez Sabel, H.}, “Pygmalion: una herramienta para la visualización interactiva del cambio semántico desde una perspectiva diacrónica”. \textit{Scire Vias: Humanidades Digitales y conocimiento. V congreso de la Sociedad Internacional de Humanidades Digitales Hispánicas (HDH 2021)}. October 4-10, 2021. Santiago de Compostela (Spain).\\
\years{2021}\underline{Bermúdez Sabel, H.}, “Multi-format publishing and re-purposing of historical linguistics data”.  \textit{EADH2021: Interdisciplinary perspectives on data. Second international conference of the European Association for Digital Humanities}. September 21-24, 2021. Siberian Federal University, Krasnoyarsk (Russia).\\
\years{2021}Dell’Oro, F., \underline{Bermúdez Sabel, H.} \amper{} P. Marongiu. “\textit{Pygmalion}, una herramienta digital para elaborar mapas semánticos: algunos casos de uso en el aula | \textit{Pygmalion}, a tool to draw interactive diachronic semantic maps: some use cases for the classroom”. \textit{I Jornada de Lexicografía en el contexto del aprendizaje de lenguas}. May 10, 2021. Universidad Complutense de Madrid, Madrid (Spain).\\
\years{2020}Dell’Oro, F. \amper{} \underline{H. Bermúdez Sabel}, “L’étude de la modalité dans un corpus diachronique en latin : théorie de la modalité, annotation linguistique et partage des données”. \textit{11\textsuperscript{e} Journée de Linguistique Suisse}. November 6, 2020. Université de Fribourg. Fribourg (Switzerland).\\
\years{2020}Ruiz Fabo, P. \amper{} \underline{H. Bermúdez Sabel}, “Rhyme network analysis in a non-canonical corpus of sonnets in Spanish”, \textit{DH2020}. July 22-24, 2020. Ottawa (Canada).\\
\years{2019}\underline{Bermúdez Sabel, H.}, Martínez Cantón, C. \amper{} P. Ruiz Fabo, “DISCOvering Spanish Sonnets: A close/distant reading experience”, \textit{Plotting Poetry (and Poetics) 3}. September 26–27, 2019. Nancy (France).\\
\years{2019}Díez Platas, M.L., \underline{Bermúdez, H.}, Ros, S., González-Blanco, E., de la Rosa, J., Pérez, A. \amper{} B. Sartini, “Una red de ontologías para la poesía europea”, \textit{IV Congreso Internacional de la Asociación de Humanidades Digitales Hispánicas}. October 23-25, 2019. Toledo (Spain).\\
\years{2019}\underline{Bermúdez Sabel, H.}, Díez Platas, M.L., Ros Muñoz, S. \amper{} E. González-Blanco, “Towards a Common Model for European Poetry: Challenges and Solutions”, \textit{DH2019}. July 9-12, 2019. Utrech (Netherlands).\\
\years{2018}\underline{Bermúdez Sabel, H.}, “Datos abertos conectados e o enriquecemento de córpora lingüísticos”, \textit{XII Congreso Internacional da AIEG}. September 18--22, 2018. Madrid (Spain).\\
\years{2018}Ruiz Fabo, P., \underline{Bermúdez Sabel, H}., Martínez Cantón, C., González-Blanco, E. \amper{} B. Navarro Colorado, “The Diachronic Spanish Sonnet Corpus (DISCO): TEI and Linked Open Data Encoding, Data Distribution and Metrical Findings”, \textit{DH2018: Puentes/Bridges.} June 26--29, 2018. Ciudad de México (Mexico). \\
\years{2018}González-Blanco, E., Ros, S., Ruiz Fabo, P., Díez Platas, M.L., \underline{Bermúdez, H.}, Martínez Cantón, C.I. \amper{} L. Ayciriex, “Poetry and Digital Humanities making interoperability possible in a divided world of digital poetry: POSTDATA project”, \textit{EADH 2018: “Data in Digital Humanities”}. December 7--9, 2018. National University of Ireland, Galway (Ireland). \\
\years{2017}\underline{Bermúdez Sabel, H.}, “Poesía, interoperabilidad y estándares para el tratamiento de datos poéticos”, \textit{Humanidades Digitales Hispánicas. III Congreso Internacional}. October 18--20, 2017. Málaga (Spain).\\
\years{2017}\underline{Bermúdez Sabel, H.}, “Anotación multicapa a distancia e o enriquecemento de edicións dixitais”, \textit{Congreso Internacional Humanidades Dixitais: olladas desde a Idade Media}. Santiago de Compostela (Spain).\\
\years{2016}\underline{Bermúdez Sabel, H.}, “Tomayto, tomahto? Encoding variant taxonomies in TEI”, \textit{TEI Conference and Members’ meeting 2016}. September 26--30, 2016. Viena (Austria).\\
\years{2015}\underline{Bermúdez Sabel, H.}, “Collatio informatizada e marcação: proposta metodológica para o estudo da variação linguística na lírica galego-portuguesa”, \textit{Congresso de Humanidades Digitais em Portugal}. October 8-9, 2015. Lisboa (Portugal).\\
\years{2015}\underline{Bermúdez Sabel, H.}, “Colación interlineal aplicada al estudio de la variación lingüística en la lírica gallego-portuguesa”, \textit{Humanidades Digitales Hispánicas. II Congreso Internacional.} October 5--7, 2015. Madrid (Spain).\\
\years{2015}Pousada Cruz, M. \amper{} \underline{H. Bermúdez Sabel}, “Cartografía literaria de la lírica profana gallego-portuguesa”, \textit{Humanidades Digitales Hispánicas. II Congreso Internacional}. October 5-7, 2015. Madrid (Spain).\\
\years{2015}\underline{Bermúdez Sabel, H.}, “Aproximação à língua dos cancioneiros galego-portugueses. Um estudo sobre a variação linguística”, \textit{XVI Congresso da Asociación Hispánica de Literatura Medieval}. September 25, 2015. Porto (Portugal). \\
\years{2015}\underline{Bermúdez Sabel, H.}, “Trovadores de corte em corte. As viagens dos trovadores galego-portugueses”, \textit{Congreso Internacional Viaxeiros: Do Antigo ao Novo Mundo}. June 3-13, 2015. Santiago de Compostela (Spain).\\
\years{2014}\underline{Bermúdez Sabel, H.}, “Linguistic variation and manuscript transmission. A case study using XML/TEI”, \textit{El’Manuscript-14}. September 15--20, 2014. Varna (Bulgaria).\\
\years{2014}\underline{Bermúdez Sabel, H.}, “Edicións filolóxicas dixitais e marcado enriquecido (XML/TEI)”, \textit{Editing for minorities in the digital era}. Santiago de Compostela (Spain).\\
\years{2013}\underline{Bermúdez Sabel, H.}, “Propuesta de edición digital para el estudio de la variación lingüística en la lírica profana gallego-portuguesa.“ \textit{Tercer Congreso Internacional Tradición e innovación: nuevas perspectivas para la edición y el estudio de documentos antiguos}. June 5--7, 2013. Salamanca (Spain).\\
\years{2012}\underline{Bermúdez Sabel, H.}, “A lírica profana galego-portuguesa: lingua e transmisión manuscrita”, \textit{Encontro da Mocidade Investigadora}. December 13--15, 2012. Santiago de Compostela (Spain).\\
\years{2012}\underline{Bermúdez Sabel, H.}, “A edición sinóptica: unha ferramenta metodolóxica para o estudo da variación lingüística aplicada á lírica profana galego-portuguesa”, \textit{X Congreso Internacional da AIEG}. September 12--14, 2012. Cardiff University (United Kingdom). 

\subsubsection*{Posters}
\noindent
\years{2022}Ruiz Fabo, P. \amper{} \underline{H. Bermúdez Sabel}, “Feature structures for character social variable annotation and an application to Alsatian theater”. \textit{TEI Conference and Members' Meeting 2022}. September 12–16, 2022. Newcastle (United Kingdom).\\
 \years{2020}Dell’Oro, F., Rimaz, L. \amper{} \underline{H. Bermúdez Sabel}, “Create your own interactive diachronic semantic maps: a flexible and user-friendly open-source tool for historical linguistics”. \textit{11\textsuperscript{e} Journée de Linguistique Suisse}. November 6, 2020. Université de Fribourg. Fribourg (Switzerland). \\
\years{2020}\underline{Bermúdez Sabel, H.}, Dell’Oro, F. \amper{} P. Marongiu, “Visualization of semantic shifts: the case of modal markers”, \textit{DH2020}. July 22-24, 2020. Ottawa (Canada).\\
\years{2020}Cayless, H., Scholger, M., \underline{Bermúdez Sabel, H.}, Meneses, L., del Rio Riande, G., Nagasaki, K., “Communicating the TEI Across Linguistic and Cultural Boundaries”, \textit{DH2020}. July 22-24, 2020. Ottawa (Canada).\\
\years{2019}\underline{Bermúdez Sabel, H.}, Martínez Cantón, C.I., Ruiz Fabo, P. \amper{} P. Plecháč, “DISCOver. Una propuesta circular para descubrir la poesía”, \textit{IV Congreso Internacional de la Asociación de Humanidades Digitales Hispánicas}. October 23-25, 2019. Toledo (Spain).\\
\years{2019}Díez Platas, M.L., \underline{Bermúdez Sabel, H.}, Ros Muñoz, S., González-Blanco, E., De La Rosa, J., Pérez Pozo, A. \amper{} L. Ayciriex, “Towards an Ontology for European Poetry”, \textit{DARIAH Annual Event 2019: Humanities Data}. September 15-17, 2019. Warsaw (Poland).\\
\years{2018}González-Blanco, E., Ros, S., Diez Platas, M. L., Ruiz Fabo, P., \underline{Bermúdez Sabel, H}., Ayciriex, L., \amper{} C. Martínez Cantón, “Poetry Lab”, \textit{CLARIN Annual Conference}. October 8-10, 2018. Pisa (Italy).\\
\years{2015}\underline{Bermúdez Sabel, H.}, “Using Feature Structures for the study of Medieval manuscripts”, \textit{DH2015 Global Digital Humanities}. June 29-July 3, 2015. Sydney (Australia). \\

\subsection{Invited talks, lectures \& workshops}
\noindent
\years{2022}Dell’Oro, F., \underline{Bermúdez Sabel, H.} \amper{} P. Marongiu, “The annotation of the main participant(s) in the modal passages of the WoPoss corpus of Latin: statistics, correlations, diachrony”. \textit{Third workshop on Modality and corpora}. December 6, 2022. University of Neuchâtel. Neuchâtel (Switzerland).\\
\years{2022}\underline{Bermúdez Sabel, H.}, “A filologia digital: novas perspetivas e desafios”, \textit{Simposio ILG 2022.
A edición dixital de textos antigos: modelos, proxectos e ferramentas}. November 28–December 2, 2022. Santiago de Compostela (Spain).\\ 
\years{2022}\underline{Bermúdez Sabel, H.}, “La
filología digital: nuevas metodologías (nuevos problemas)”. \textit{II IRCVM International Conference: Digitizing the Middle Ages}. Medieval Cultures Research Institute (IRCVM). October 5--7, 2022. Barcelona (Spain).\\
\years{2022}\underline{Bermúdez Sabel, H.} \amper{} P. Ruiz Fabo, “Análisis de sentimientos aplicado a la rima
en un corpus histórico”. \textit{Textos hispánicos y humanidades digitales:
perspectivas actuales. Seminario Internacional}. June 22--24, 2022. Universidad de La Laguna. La Laguna (Spain).\\
\years{2022}\underline{Bermúdez Sabel, H.} \amper{} C. Montrichard, “Présentation du projet: Les corpora latins et français~: une fabrique pour l’accès à la représentation des connaissances”. Lecture for the  MA seminar \textit{Linguistique de corpus} (Prof. Corinne Rossari). May 30, 2022. Université Neuchâtel. Neuchâtel (Switzerland).\\
\years{2022}Dell’Oro, F., \underline{Bermúdez Sabel, H.} \amper{} P. Marongiu, “L’annotation de corpus: perspectives synchroniques et diachroniques”. Lecture for the  MA seminar \textit{Linguistique de corpus} (Prof. Corinne Rossari). April 4, 2022. Université Neuchâtel. Neuchâtel (Switzerland).\\
\years{2022}\underline{Bermúdez Sabel, H.}, “Edición digital y lingüística histórica”. Section d’espagnol. Université de Lausanne. March 18, 2022. Lausanne (Switzerland).\\
\years{2021}López Izquierdo. M., Taillot, A., Yusta Rodrigo, M., \underline{Bermúdez Sabel, H.} \amper{} Studio Atlantis, “La carta como encrucijada: el proyecto CAREXIL-FR”. \textit{Epistolâtries: mutations contemporaines et nouvelles approches d'étude de la lettre}. December 2-3, 2021. Université Paris 8 /  Université Paris Nanterre, Paris (France).\\
\years{2021}Dell’Oro, F. \amper{} \underline{H. Bermúdez Sabel}, “La modalité dans les Évangiles (latin et grec)”. \textit{Linguistique historique et linguistique de corpus : perspectives syntaxiques, sémantiques et pragmatiques avec un regard sur la modalité}. Programme doctoral en Sciences du langage de la Conférence universitaire de Suisse occidentale (CUSO). October 18-19, 2021. Les Diablerets (Switzerland). \\
\years{2021}Dell’Oro, F., \underline{Bermúdez Sabel, H.}, Marongiu, P. \amper{} L. Rimaz, \textit{Pygmalion, un Outil Lexicographique pour dessiner des Cartes Interactives}. May 11, 2021. Université de Neuchâtel. Neuchâtel (Switzerland).\\
\years{2021}\underline{Bermúdez Sabel, H.}, “Comment construire un corpus pour l’analyse en linguistique historique”. Lecture for the MA seminar \textit{Linguistique de corpus} (Prof. Corinne Rossari). April 19, 2021. Université de Neuchâtel. Neuchâtel (Switzerland).\\
 \years{2020}\underline{Bermúdez Sabel, H.}, “El modelado de fenómenos lingüísticos en TEI: propuestas de etiquetado y de su explotación”, \textit{Annotation et interaction dans les corpus numériques: le fonds épistolaire de CAREXIL-FR}. November 4, 2020. Université Paris 8 Vincennes Saint-Denis, Paris (France).\\
 \years{2020}\underline{Bermúdez Sabel, H.}, “Metodologías para la visualización de datos en Humanidades”. Lecture for the MA course \textit{Metodologías para la investigación científica en el ámbito de las Humanidades Digitales}. Máster Universitario en Humanidades Digitales. March 16th, 2020. Universidad Internacional de la Rioja (UNIR).\\
\years{2019}Rio Riande, G. del, Scholger, M., \underline{Bermúdez Sabel, H.}, Nagasaki, K., Meneses, L. \amper{} H. Cayless, “Communicating the TEI to a Multilingual User Community”, \textit{Scholarly Communication Institute}. October 13--17, 2019. Chapel Hill, N.C. (USA).\\
\years{2019}\underline{Bermúdez Sabel, H.} \amper{} F. Dell’Oro, “Une édition numérique pour explorer un corpus de (retro-)traductions parallèles”, \textit{Humanités numériques et texte littéraire traduit. Editer et analyser des corpus de versions parallèles}. September 24, 2019. Université Grenoble Alpes, Grenoble (France).\\
\years{2019}\underline{Bermúdez Sabel, H.}, “Standardization of European Poetry”, \textit{Pozvánja na prednášky a workshop}. March 15, 2019. Univerzita Konštantína Filosofa v Nitre, Nitra (Slovaquia).\\
\years{2019}\underline{Bermúdez Sabel, H.} \amper{} C. Martínez Cantón, “DISCO. An Interface to Browse a Spanish Sonnet Corpus”, \textit{Autorské korpusy a jejich vyzužití v literární vĕdĕ} (Authorial corpora and their usefulness in Literary studies). March 5, 2019. Palacký University Olomouc, Olomouc (Czech Republic).\\
\years{2019}Rojas, A. \amper{} \underline{H. Bermúdez Sabel}, “Exercices pratiques d’encodage”, \textit{Atelier d’initiation à l’enco-dage XML-TEI et à la fouille de textes en espagnol}, Projet CLEA (EA 4083). January, 21--25, 2019. Sorbonne Université, Paris (France).\\
\years{2018}\underline{Bermúdez Sabel, H.}, “Lenguajes de consulta XML para corpus anotados lingüísticamente”, \textit{Aplicaciones y posibilidades del procesamiento del lenguaje natural para la investigación en Humanidades}. Cursos de Verano UNED. July 9--11, 2018. Madrid (Spain).\\
\years{2018}Ruiz Fabo, P. \amper{} \underline{H. Bermúdez Sabel}, \textit{POSTDATA: Poetry Standardization and Linked Open Data}. June 18--21, 2018. Trinity College Dublin. Dublin (Ireland).\\
\years{2018}\underline{Bermúdez Sabel, H.}, “La filología y el texto digital”, \textit{VIII Jornadas Digitales “Edición académica: el entorno digital y sus retos”,} Unión de Editoriales Universitarias Españolas. June 7--8, 2018. Madrid (Spain).\\
\years{2018}\underline{Bermúdez Sabel, H.} \amper{} P. Ruiz Fabo, \textit{Linked Open Data: Unchain your corpora}. April 25, 2018. University of Würzburg. Würzburg (Germany).\\
\years{2018}\underline{Bermúdez Sabel, H.}, “Towards interoperability in the European poetry community”, \textit{Shaping Data in Digital Humanities}. Centre for Communication and Computing (University of Copenhagen). April 20, 2018. Copenhagen (Denmark). \\
\years{2017}\textit{Introduction to programming and mark-up languages}, PhD Training Program for candidates in the Humanities, Universidade de Santiago de Compostela. June 3--5, 2017. Santiago de Compostela (Spain).\\ %falta 
\years{2017}Curado Malta, M. \amper{} \underline{H. Bermúdez Sabel}, “Un modelo de datos para la poesía en el contexto de los datos enlazados”, Tecnologías semánticas y herramientas lingüísticas para Humanidades Digitales. Cursos de verano UNED. July 5--7, 2017. Madrid (Spain).\\
\years{2017}Curado Malta, M. \amper{} \underline{H. Bermúdez Sabel}, “Building the Domain Model”, \textit{Building a common model for semantic interoperability in the digital poetry ecosystem}. POSTDATA workshop. March 15--17, 2017. Madrid (Spain). \\
\years{2016}Gamallo, P. \amper{} \underline{H. Bermúdez Sabel}, \textit{Introduction to programming and mark-up languages}, PhD Training Program for candidates in the Humanities, Universidade de Santiago de Compostela (Spain).\\ %falta 
\years{2016}\underline{Bermúdez Sabel, H.}, “Tecnologías de marcado específicas para poesía: TEI-XML”, \textit{Tecnologías digitales aplicadas al estudio de la poesía}. Cursos de verano UNED. June 27--July 1, 2016. Madrid (Spain). \\
\years{2016}\underline{Bermúdez Sabel, H.}, “Hermenéutica dixital, métodos computacionais e filoloxía”, \textit{Retazos da arte, sociedade e cultura na Idade Media Europea III: Das Humanidades Dixitais a J. R. R. Tolkien, novos horizontes nos estudos medievais}. Universidade de Verán – Universidade de Santiago de Compostela. June 8--10, 2016. Santiago de Compostela (Spain).\\
\years{2016}Birnbaum, David J. \amper{} \underline{H. Bermúdez Sabel}, “Digital collation tools”, \textit{Text as process: Genetic and Textual Criticism in the Digital Age}. April 4--6, 2016. University of Pittsburgh (USA).\\
\years{2016}\underline{Bermúdez Sabel, H.}, “An orientation to Zotero and LaTeX”, Lecture to the English Literature Capstone course at the University of Pittsburgh at Greensburg. February 2, 2016. Greensburg PA (USA).\\
\years{2015}\underline{Bermúdez Sabel, H.}, “Tecnologías XML y análisis de redes. Vinculación entre centros de producción literaria y centros de poder”, \textit{Espiritualidad en la Edad Media. Perspectivas y Metodologías}. Proyecto Claustra. December 14--15, 2015. Santiago de Compostela (Spain).\\
\years{2015}\underline{Bermúdez Sabel, H.}, \textit{Tecnoloxías X para medievalistas}. Seminario de Estudios Medievales Hispánicos. 5, 2015. November Santiago de Compostela (Spain).\\
\years{2014}Birnbaum, D., Bojadžiev, A. \amper{} \underline{H. Bermúdez Sabel}, \textit{XML and TEI for Slavic Philology}. September 15--20, 2014. Varna (Bulgaria).



\subsection{Digital Humanities Projects}\label{DHprojects}

\subsubsection{Databases with Graphical User Interfaces}

\years{2019}\underline{Bermúdez Sabel, H.} \textit{Galician-Portuguese secular lyric: philology and historical linguistics}\\
\href{http://gl-pt.obdurodon.org}{http://gl-pt.obdurodon.org}
\begin{itemize}[noitemsep,topsep=0pt]
\item Role: Principal investigator. Individual research project.
  \item Development of a synoptic digital edition.
  \item Transcription of medieval texts encoding palaeographic particularities.
  \item Development of a digital workstation for textual exploration and quantitative analysis of linguistic variation phenomena.
\item Technologies and software employed: XML-TEI, XSLT, XQuery, SVG, PHP, JavaScript, eXist DB.
 \end{itemize}

\years{2019}\underline{Bermúdez Sabel, H.}, Martínez Cantón, C. \amper{} P. Ruiz Fabo. \textit{DISCOver: an interface to explore the Diachronic Spanish Sonnet Corpus (DISCO)}\\
\href{http://prf1.org/disco}{http://prf1.org/disco}
\begin{itemize}[noitemsep,topsep=0pt]
\item Role: Graphical User Interface main developer.
\item Technologies employed: SQL, PHP, JavaScript.
\end{itemize}


\subsubsection{Datasets with Graphical User Interfaces}

\years{2021}\underline{Bermúdez Sabel, H.} \textit{Édition hyperDIplomatique de la lyrique GAlégo-portugaise (DIGA)}.\\
\href{https://helenasabel.github.io/DIGA/}{https://helenasabel.github.io/DIGA/}

\begin{itemize}[noitemsep,topsep=0pt]
\item Role: Principal investigator. Individual research project.
\item Text encoding of the paleographic transcriptions, design and development of the
website and its functionalities.
\item Technologies employed: XSLT, JavaScript.
\end{itemize}


\years{2020}\underline{Bermúdez Sabel, H.} \amper{} F. Dell’Oro. \textit{Une édition numérique pour explorer un corpus de (retro-)traductions parallèles}.\\
\href{https://woposs.unine.ch/tradnum/}{https://woposs.unine.ch/tradnum}

\begin{itemize}[noitemsep,topsep=0pt]
\item Role: Conceptualization of the edition. Modeling and text encoding of the texts. Design of the functionalities and
developer of the GUI.
\item Technologies employed: XSLT, JavaScript.
\end{itemize}


\years{2017}Triplette, S., Beshero-Bondar, E. \amper{} \underline{H. Bermúdez
Sabel}. \textit{Amadis in Translation}\\
\href{https://newtfire.org/amadis}{https://newtfire.org/amadis}

\subsubsection{Software}
\years{2020}Dell’Oro, F., Rimaz, L., \underline{Bermúdez Sabel, H.} \amper{} P. Marongiu. \textit{Pygmalion. A tool to draw interactive semantic maps}\\
\href{https://woposs.unine.ch/pygmalion.html}{https://woposs.unine.ch/pygmalion.html}

\begin{itemize}[noitemsep,topsep=0pt]
 \item Role: conceptualisation, assisting with development, testing.
 \item Technologies employed: JavaScript, SVG.
\end{itemize}


\subsubsection{Pedagogical materials}
\urlstyle{same}
\noindent
\years{2020}\underline{Bermúdez Sabel, H.} \amper{} F. Dell’Oro. \textit{Automatic annotation of classical languages: Greek and Latin}. 
\url{https://github.com/WoPoss-project/automatic\_annotation} \\
\years{2020}\underline{Bermúdez Sabel, H.}, Nury, E. \amper{} E. Spadini. \textit{Introduction to automatic collation}. \url{https://automaticcollationlausanne2020.github.io}\\
\years{2015}\underline{Bermúdez Sabel, H.} \textit{Introduction to KML}. \href{http://dh.obdurodon.org/kml/kml-tutorial.xhtml}{http://dh.obdurodon.org/kml/kml-tutorial.xhtml}

\subsubsection{Data visualization}

\years{2016}\textit{Mapping Medieval Galician-Portuguese Poetry and its networks}\\
\href{http://www.usc.es/athene}{http://www.usc.es/athene}

\begin{itemize}[noitemsep,topsep=0pt]
\item Role: Principal investigator. Individual research project.
 \item Development of interactive cartographic representations concerning Galician-Portuguese troubadours’ biographical data and the cultural centres of this poetic school.
 \item Network analysis and visualization.
 \item Technologies and software employed: XML-TEI, XSLT, KML, SVG, Cytoscape, Google Earth.
\end{itemize}


\section{Participation in research teams}
% \years{2020--2022}\textit{Communicating the TEI to a Multilingual User Community} (2001-07353). Funded by The Andrew W. Mellon Foundation.\\
\years{2022--2024}\textit{MÁS POEsía para MÁS gente. La poesía en la música popular española contemporánea
(PoeMAS)} (PID2021-125022NB-I00). Funded by the Ministerio de Ciencia, Innovación y Universidades (Spanish Ministry of Science, Innovation and Universities). PIs: Clara Martínez Cantón \amper{} Guillermo Laín (Universidad Nacional de Educación a Distancia, Spain).\\
\years{2022-2023}\textit{Les corpora latins et français : une fabrique pour l’accès à la représentation des connaissances}. Funded by the Empiris Foundation (Switzerland). PIs: Francesca Dell’Oro \amper{} Corinne Rossari (Université de Neuchâtel, Switzerland). \\
\years{2021-2024}\textit{Cancioneros gallego-portugueses. De la Paleografía digital a la Gramática histórica} (PID2020-113491GB-I00). Funded by the Spanish Ministry of Science and Education. PI: Pilar Lorenzo Gradín (Universidade de Santiago de Compostela, Spain).\\
\years{2020--}\textit{CARtas de REpublicanos Españoles REfugiados y EXILiados en FRancia (CAREXIL-FR)}. PI: Marta López Izquierdo (Université Paris 8, France).\\
\years{2019--2023}\textit{A world of possibilities. Modal pathways over an extra-long period of time: the diachrony of modality in the Latin language} (SNSF n° PP00P1\_176778). Funded by the Swiss National Foundation. PI: Francesca Dell’Oro (Université de Neuchâtel, Switzerland).\\
\years{2019--2021}\textit{POEsía para MÁS gente. La poesía en la música popular española contemporánea
(PoeMAS)} (PGC2018-099641-A-I00). Funded by the Ministerio de Ciencia, Innovación y Universidades (Spanish Ministry of Science, Innovation and Universities). PIs: Clara Martínez Cantón \amper{} Guillermo Laín (Universidad Nacional de Educación a Distancia, Spain).\\
\years{2017--2019}\textit{POSTDATA – Poetry Standardization and Linked Open Data} (ERC-2015-STG-679528). 
Funded by the European Research Council (ERC). PI: Elena González-Blanco (Universidad Nacional de Educación a Distancia, Spain).\\
\years{2016--2019}\textit{Paleografía, lingüística y filología. Laboratorio online de la lírica gallego-portuguesa} (FFI2015-68451-P). Funded by the Ministerio de Economía y Competitividad (Spanish Ministry of Economy and Competitiveness). PIs: Mercedes Brea \amper{} Pilar Lorenzo Gradín (Universidade de Santiago de Compostela, Spain).\\
\years{2014--2018}\textit{Interdisciplinary Medieval Studies Network} (2016-PG069 / 2014-PG134). Funded by the Xunta de Galicia (Autonomous Community of Galicia Government). PI: Mercedes Brea (Universidade de Santiago de Compostela, Spain).\\
\years{2013--2016}\textit{Competitive research unit GI-1350-Románicas} (GRC 2013-046). Funded by the Xunta de Galicia (Autonomous Community of Galicia Government). PI: Mercedes Brea (Universidade de Santiago de Compostela, Spain).


\section{Teaching}

\subsection{Digital Humanities courses}

\years{2018--2022}\textit{TEI Mark-up and Annotation (II): XSLT, XPath XQuery (Transformations)}. 
\ind{Role:} responsible. 
\ind{Hours:} 5 ECTS credits.
\ind{Institution:} National Distance Education University (UNED).
\ind{Degree:} Master in Digital Humanities.\\
\years{2016--2022}\textit{Close and distant reading: visualizing data in the Humanities}.
\ind{Hours:} 5 ECTS credits. 
\ind{Institution:} National Distance Education University (UNED). 
\ind{Degree:} Master in Digital Humanities / Professional Certificate in Digital Edition / Professional Certificate in Digital Humanities.\\
\years{2016--2017}\textit{TEI modules: transcription of primary sources, manuscripts, and verse}. 
\ind{Role:} responsible. 
\ind{Hours:} 5 ECTS credits. 
\ind{Institution:} National Distance Education University (UNED). 
\ind{Degree:} Professional Certificate in Digital Edition.\\
\years{2015--2016}\textit{Computational Methods in the Humanities}. 
\ind{Role:} teaching assistant. 
\ind{Hours:} 6 credits (\~{} 12 ECTS). 
\ind{Institution:} University of Pittsburgh.\\
%\ind{Degree:} Degrees in Humanities (Honors College)\\
      \years{2015--2017}\textit{Information Technologies in Romance linguistic and literary studies}. 
\ind{Role:} practical seminars. 
\ind{Hours:} 2 of 6 ECTS credits. 
\ind{Institution:} Universidade de Santiago de Compostela. 
\ind{Degree:} Major in Romance Philology – Second year.\\
      
\subsection{Linguistics, Romance Philology \&  Medieval Studies}
\years{2022}\textit{Introduction to Historical Linguistics}. 
\ind{Responsible:} Francesca Dell’Oro. 
\ind{Role:} teaching assistant. 
\ind{Hours:} 3 of 28 hours. 
\ind{Institution:} Université de Neuchâtel. 
\ind{Degree:} BA (language sciences itinerary).\\
\years{2016--2017}\textit{The emergence of Romance languages}\\
\ind{Role:} practical seminars.
\ind{Hours:} 1 of 6 ECTS credits. 
\ind{Institution:} Universidade de Santiago de Compostela. 
\ind{Degree:} Major in Romance Philology – Second year.\\
\years{2016--2017}\textit{Art and Literature in the Ancient and Medieval Worlds}. 
\ind{Role:} practical seminars. 
\ind{Hours:} 2 of 6 ECTS credits. 
\ind{Institution:} Universidade de Santiago de Compostela. 
\ind{Degree:} Major in Art History – First year. \\
\years{2015--2016}\textit{Transmission of Romance texts}.
\ind{Role:} practical seminars. 
\ind{Hours:} 2 of 6 ECTS credits. 
\ind{Institution:} Universidade de Santiago de Compostela. 
\ind{Degree:} Major in Romance Philology – Second year. \\
\years{2015--2016}\textit{Romance Literature: short narrative}. 
\ind{Role:} practical seminars. 
\ind{Hours:} 2 of 6 ECTS credits. 
\ind{Institution:} Universidade de Santiago de Compostela. 
\ind{Degree:} Major in Romance Philology – Fourth year. \\
\years{2014--2015}\textit{Romance Textual Criticism}. 
\ind{Role:} practical seminars. 
\ind{Hours:} 2 of 6 ECTS credits. 
      \ind{Institution:} Universidade de Santiago de Compostela. 
\ind{Degree:} Major in Romance Philology – Fourth year.

\subsection{Student supervision}

\ind{Student:} Laura Dobrita\\
\ind{PhD thesis title:} \textit{La incorporación y adaptación del soneto a la versificación rumana. Características
métricas de los primeros sonetos en rumano (1810-1914)}\\
\ind{Role:} Co-director (Director: Clara Isabel Martínez Cantón)\\
\ind{Degree:} Programa de Doctorado en Filología. Estudios lingüísticos y literarios\\
\ind{Institution:} Universidad Nacional de Educación a Distancia (Spain)\\
\ind{Estimated date of defense:} February 2023\\

\ind{Student:} Caroline Müller\\
\ind{Master thesis title:} \textit{The semi-automatic markup of lexical attestations in source texts for the Diccionario del Español
Medieval electrónico}\\
\ind{Role:} Main supervisor\\
\ind{Degree:} Diploma de Especialista Universitario en Humanidades Digitales\\
\ind{Institution:} Universidad Nacional de Educación a Distancia (Spain)\\
\ind{Date of defense:} 23/08/2022\\


\ind{Student:} Carmen Grijalba Pena\\
\ind{Master thesis title:} \textit{Visualización de datos a partir de las colección diplomática de Sancho III Garcés el Mayor (1004--1035)}\\
\ind{Role:} Main supervisor\\
\ind{Degree:} Diploma de Especialista Universitario en Humanidades Digitales\\
\ind{Institution:} Universidad Nacional de Educación a Distancia (Spain)\\
\ind{Date of defense:} 09/08/2022\\

% \ind{Student:} Alejandra Grandal Castillo\\
% \ind{Master thesis title:} \textit{El género en las novelas de Jane Austen}\\
% \ind{Role:} Main supervisor\\
% \ind{Degree:} Diploma de Especialista Universitario en Humanidades Digitales\\
% \ind{Institution:} Universidad Nacional de Educación a Distancia (Spain)\\
% \ind{Estimated date of defense:} September 2021\\

\ind{Student:} Filomena Anna Dalessandro\\
\ind{Master thesis title:} \textit{Emigración femenina en la literatura contemporánea: Italia y España}\\
\ind{Role:} Main supervisor\\
\ind{Degree:} Diploma de Especialista Universitario en Humanidades Digitales\\
\ind{Institution:} Universidad Nacional de Educación a Distancia (Spain)\\
\ind{Date of defense:} 09/09/2020

\section{Membership in scientific committees and reviewing activities}
\subsection{Participation in technical and scientific committees}
\years{2022--}Chair of the Stylesheets Group of the TEI Technical Council.\\
\years{2021--}Elected member of the Text Encoding Initiative's (TEI) Technical Council.\\
\years{2020--}Member of the Text Encoding Initiative Working Group “Internationalization (I18n)”.\\
\years{2022}Member of the scientific committee of the international conference \textit{Simposio ILG 2022.
A edición dixital de textos antigos: modelos, proxectos e ferramentas} (November 28--December 2, 2022, Instituto da Lingua Galega).\\
\years{2022}Member of the jury (as external expert) for the MA thesis \textit{La tecnología al servicio de la} collatio codicum\textit{: el caso del Fuero Juzgo} by Charles Mabille (University of Lausanne, Switzerland).\\
\years{2022}Member of the scientific committee of the international conference \textit{Machine Learning
and Data Mining for Digital Scholarly Editions} (June 9-10, 2022, University of Rostock).\\
\years{2021}Member of the scientific committee of the international conference \textit{Epistolâtries : mutations contemporaines et nouvelles approches d’étude de la lettre} (December 2--3, 2021, Université Paris 8 / Université Paris Nanterre).\\
\years{2020}Member of the scientific committee of the conference \textit{TheorLing: Theoretical linguistics in the light of the interaction of qualitative and quantitative approaches} (June 21-22, 2021, University of Neuchâtel).\\
\years{2019}Ph.D. Thesis Assesment for International Mention of the doctoral thesis \textit{Gestualidad y acción femeninas en espejos de damas del siglo XV} by Laura Pereira Domínguez (Universidade de Santiago de Compostela).\\
\years{2019}Technical reports for the publishing house Editorial de la Universidad de Sevilla.\\

\subsection{Reviewer}
\years{2022}Reviewer of a monograph for Heidelberg University Publishing (heiUP).\\
\years{2022}Reviewer of one article submitted to: \textit{Journal of Data Mining \amper{} Digital Humanities}.\\
\years{2021}Reviewer of one contribution submitted to: Sánchez Cabrera, M. \amper{} Mármol Ávila, P. (eds.): \textit{Nuevas tecnologías e innovación docente en literatura hispánica}. Madrid: Sial Pigmalión.\\
\years{2020}Reviewer of one article submitted to: \textit{Bulletin of Hispanic Studies} (ISSN: 1475-3839).\\
\years{2020}Reviewer of one contribution submitted to: Spadini, E., Tomasi, F., \amper{} Vogeler, G. (2021) (eds.): \textit{Graph Data-Models and Semantic Web Technologies in Scholarly Digital Editing}. BoD – Books on Demand.\\
\years{2020}Reviewer for the conference \textit{Digital Humanities 2020 -
Intersections/Carrefours}, Ottawa, Canada.\\
\years{2019}Reviewer of one article submitted to: \textit{Liinc em Revista} (ISSN: 1808-3536).\\
\years{2018--2020}Reviewer of one article per issue for the journal \textit{Revista de Humanidades Digitales} (ISSN: 2531-1786).

%\hrule
\section{Organization of scientific events}
\years{2022}Member of the Organising Committee for the International Conference \textit{Modality in historical stages: methods and resources for investigating modality}. November 14--15, 2022. Neuchâtel (Switzerland).\\
\years{2017}Member of the Organising Board for the International Conference \textit{Humanidades Dixitais: olladas cara á Idade Media}, organised by the Rede de Estudos Medievais Interdisciplinares. October 9--11, 2017. Santiago de Compostela (Spain).\\
\years{2015}Member of the Local Organising Committee of the International Colloquium \textit{A expresión das emocións na lírica románica medieval}, organised by the Area of Romance Philology of the Universidade de Santiago de Compostela. March 10--12, 2015. Santiago de Compostela (Spain). \\
\years{2012}Member of the Local Organising Committee of the International Colloquium \textit{Parodia e debate metaliterarios na Idade Media}, organised by the Area of Romance Philology of the Universidade de Santiago de Compostela and the Asociación Hispánica de Literatura Medieval. November 7--8, 2012. Santiago de Compostela (Spain).


\section{Prizes, awards \& fellowships}
\noindent
\years{2018}\textit{Josef Dobrovský Fellowship} funded by the Czech Academy of Sciences. 10/2018.\\
\years{2017} Best Student Paper Award, \textit{Language, Data  and Knowledge 2017}. June 19--20, 2017. Galway, Ireland.\\
\years{2017} Best Datathon Result Award, \textit{Second Datathon on Linguistic Linked Open Data}. June 26--30, 2017. Cercedilla, Spain.\\
\years{2013--2016}\textit{PhD Scholarship (University Faculty Training Programme)} funded by the Ministerio de Educación, Cultura y Deporte (Spanish Government).\\
\years{2014}\textit{Postgraduate grant for international study} funded by the Fundación Pedro Barrié de la Maza.\\
\years{2012--2013}\textit{Predoctoral Scholarship} funded by the Consellería de Cultura Educación e Ordenación Universitaria (Government of the Autonomous Community of Galicia).\\
\years{2012}\textit{Premio Fin de Carreira da Xunta de Galicia} (highest GPA in the correspondent BA degree in all three universities of the Autonomous Community of Galicia).\\
\years{2011--2012}\textit{Scholarship for studies leading to master’s degree} funded by the Ministerio de Educación, Cultura y Deporte (Spanish Government).


%\vspace{1cm}
\vfill{}
%\hrulefill

\begin{center}
{\scriptsize  Last updated: \today\- %•\- Typeset in \XeTeX 
\\
%\fontspec{Times New Roman}
\href{https://www.unine.ch/isla/home/equipe/helena-bermudez-sabel.html}{https://www.unine.ch/isla/home/equipe/helena-bermudez-sabel.html}
}
\end{center}

\end{document}
